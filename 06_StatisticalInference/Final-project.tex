\documentclass[]{article}
\usepackage{lmodern}
\usepackage{amssymb,amsmath}
\usepackage{ifxetex,ifluatex}
\usepackage{fixltx2e} % provides \textsubscript
\ifnum 0\ifxetex 1\fi\ifluatex 1\fi=0 % if pdftex
  \usepackage[T1]{fontenc}
  \usepackage[utf8]{inputenc}
\else % if luatex or xelatex
  \ifxetex
    \usepackage{mathspec}
  \else
    \usepackage{fontspec}
  \fi
  \defaultfontfeatures{Ligatures=TeX,Scale=MatchLowercase}
\fi
% use upquote if available, for straight quotes in verbatim environments
\IfFileExists{upquote.sty}{\usepackage{upquote}}{}
% use microtype if available
\IfFileExists{microtype.sty}{%
\usepackage[]{microtype}
\UseMicrotypeSet[protrusion]{basicmath} % disable protrusion for tt fonts
}{}
\PassOptionsToPackage{hyphens}{url} % url is loaded by hyperref
\usepackage[unicode=true]{hyperref}
\hypersetup{
            pdftitle={Peer-graded Assignment: Statistical Inference Course Project PART1},
            pdfborder={0 0 0},
            breaklinks=true}
\urlstyle{same}  % don't use monospace font for urls
\usepackage[margin=1in]{geometry}
\usepackage{color}
\usepackage{fancyvrb}
\newcommand{\VerbBar}{|}
\newcommand{\VERB}{\Verb[commandchars=\\\{\}]}
\DefineVerbatimEnvironment{Highlighting}{Verbatim}{commandchars=\\\{\}}
% Add ',fontsize=\small' for more characters per line
\usepackage{framed}
\definecolor{shadecolor}{RGB}{248,248,248}
\newenvironment{Shaded}{\begin{snugshade}}{\end{snugshade}}
\newcommand{\KeywordTok}[1]{\textcolor[rgb]{0.13,0.29,0.53}{\textbf{#1}}}
\newcommand{\DataTypeTok}[1]{\textcolor[rgb]{0.13,0.29,0.53}{#1}}
\newcommand{\DecValTok}[1]{\textcolor[rgb]{0.00,0.00,0.81}{#1}}
\newcommand{\BaseNTok}[1]{\textcolor[rgb]{0.00,0.00,0.81}{#1}}
\newcommand{\FloatTok}[1]{\textcolor[rgb]{0.00,0.00,0.81}{#1}}
\newcommand{\ConstantTok}[1]{\textcolor[rgb]{0.00,0.00,0.00}{#1}}
\newcommand{\CharTok}[1]{\textcolor[rgb]{0.31,0.60,0.02}{#1}}
\newcommand{\SpecialCharTok}[1]{\textcolor[rgb]{0.00,0.00,0.00}{#1}}
\newcommand{\StringTok}[1]{\textcolor[rgb]{0.31,0.60,0.02}{#1}}
\newcommand{\VerbatimStringTok}[1]{\textcolor[rgb]{0.31,0.60,0.02}{#1}}
\newcommand{\SpecialStringTok}[1]{\textcolor[rgb]{0.31,0.60,0.02}{#1}}
\newcommand{\ImportTok}[1]{#1}
\newcommand{\CommentTok}[1]{\textcolor[rgb]{0.56,0.35,0.01}{\textit{#1}}}
\newcommand{\DocumentationTok}[1]{\textcolor[rgb]{0.56,0.35,0.01}{\textbf{\textit{#1}}}}
\newcommand{\AnnotationTok}[1]{\textcolor[rgb]{0.56,0.35,0.01}{\textbf{\textit{#1}}}}
\newcommand{\CommentVarTok}[1]{\textcolor[rgb]{0.56,0.35,0.01}{\textbf{\textit{#1}}}}
\newcommand{\OtherTok}[1]{\textcolor[rgb]{0.56,0.35,0.01}{#1}}
\newcommand{\FunctionTok}[1]{\textcolor[rgb]{0.00,0.00,0.00}{#1}}
\newcommand{\VariableTok}[1]{\textcolor[rgb]{0.00,0.00,0.00}{#1}}
\newcommand{\ControlFlowTok}[1]{\textcolor[rgb]{0.13,0.29,0.53}{\textbf{#1}}}
\newcommand{\OperatorTok}[1]{\textcolor[rgb]{0.81,0.36,0.00}{\textbf{#1}}}
\newcommand{\BuiltInTok}[1]{#1}
\newcommand{\ExtensionTok}[1]{#1}
\newcommand{\PreprocessorTok}[1]{\textcolor[rgb]{0.56,0.35,0.01}{\textit{#1}}}
\newcommand{\AttributeTok}[1]{\textcolor[rgb]{0.77,0.63,0.00}{#1}}
\newcommand{\RegionMarkerTok}[1]{#1}
\newcommand{\InformationTok}[1]{\textcolor[rgb]{0.56,0.35,0.01}{\textbf{\textit{#1}}}}
\newcommand{\WarningTok}[1]{\textcolor[rgb]{0.56,0.35,0.01}{\textbf{\textit{#1}}}}
\newcommand{\AlertTok}[1]{\textcolor[rgb]{0.94,0.16,0.16}{#1}}
\newcommand{\ErrorTok}[1]{\textcolor[rgb]{0.64,0.00,0.00}{\textbf{#1}}}
\newcommand{\NormalTok}[1]{#1}
\usepackage{longtable,booktabs}
% Fix footnotes in tables (requires footnote package)
\IfFileExists{footnote.sty}{\usepackage{footnote}\makesavenoteenv{long table}}{}
\usepackage{graphicx,grffile}
\makeatletter
\def\maxwidth{\ifdim\Gin@nat@width>\linewidth\linewidth\else\Gin@nat@width\fi}
\def\maxheight{\ifdim\Gin@nat@height>\textheight\textheight\else\Gin@nat@height\fi}
\makeatother
% Scale images if necessary, so that they will not overflow the page
% margins by default, and it is still possible to overwrite the defaults
% using explicit options in \includegraphics[width, height, ...]{}
\setkeys{Gin}{width=\maxwidth,height=\maxheight,keepaspectratio}
\IfFileExists{parskip.sty}{%
\usepackage{parskip}
}{% else
\setlength{\parindent}{0pt}
\setlength{\parskip}{6pt plus 2pt minus 1pt}
}
\setlength{\emergencystretch}{3em}  % prevent overfull lines
\providecommand{\tightlist}{%
  \setlength{\itemsep}{0pt}\setlength{\parskip}{0pt}}
\setcounter{secnumdepth}{0}
% Redefines (sub)paragraphs to behave more like sections
\ifx\paragraph\undefined\else
\let\oldparagraph\paragraph
\renewcommand{\paragraph}[1]{\oldparagraph{#1}\mbox{}}
\fi
\ifx\subparagraph\undefined\else
\let\oldsubparagraph\subparagraph
\renewcommand{\subparagraph}[1]{\oldsubparagraph{#1}\mbox{}}
\fi

% set default figure placement to htbp
\makeatletter
\def\fps@figure{htbp}
\makeatother


\title{Peer-graded Assignment: Statistical Inference Course Project PART1}
\author{}
\date{\vspace{-2.5em}}

\begin{document}
\maketitle

\paragraph{Gabriela Ochoa}\label{gabriela-ochoa}

\section{\texorpdfstring{\textbf{Instructions}}{Instructions}}\label{instructions}

The project consists of two parts:

\begin{itemize}
\tightlist
\item
  A simulation exercise.
\item
  Basic inferential data analysis.
\end{itemize}

\subsection{\texorpdfstring{\textbf{Part 1: Simulation Exercise
instructions}}{Part 1: Simulation Exercise instructions}}\label{part-1-simulation-exercise-instructions}

Illustrate via simulation and associated explanatory text the properties
of the distribution of the mean of 40 exponentials.

\begin{Shaded}
\begin{Highlighting}[]
\KeywordTok{hist}\NormalTok{(}\KeywordTok{runif}\NormalTok{(}\DecValTok{1000}\NormalTok{))}
\end{Highlighting}
\end{Shaded}

\includegraphics{Final-project_files/figure-latex/unnamed-chunk-1-1.pdf}

and the distribution of 1000 averages of 40 random uniforms

\begin{Shaded}
\begin{Highlighting}[]
\NormalTok{mns =}\StringTok{ }\OtherTok{NULL}
\ControlFlowTok{for}\NormalTok{ (i }\ControlFlowTok{in} \DecValTok{1} \OperatorTok{:}\StringTok{ }\DecValTok{1000}\NormalTok{) mns =}\StringTok{ }\KeywordTok{c}\NormalTok{(mns, }\KeywordTok{mean}\NormalTok{(}\KeywordTok{runif}\NormalTok{(}\DecValTok{40}\NormalTok{)))}
\KeywordTok{hist}\NormalTok{(mns)}
\end{Highlighting}
\end{Shaded}

\includegraphics{Final-project_files/figure-latex/unnamed-chunk-2-1.pdf}
This distribution looks far more Gaussian than the original uniform
distribution!

This exercise is asking you to use your knowledge of the theory given in
class to relate the two distributions.

\subsection{\texorpdfstring{\textbf{Resolution Part
1}}{Resolution Part 1}}\label{resolution-part-1}

We start by running a 1000 simulations of 40 exponentials.

\begin{Shaded}
\begin{Highlighting}[]
\NormalTok{## Required libraries}
\KeywordTok{library}\NormalTok{(ggplot2)}
\KeywordTok{library}\NormalTok{(knitr)}
\NormalTok{## setting seed}
\KeywordTok{set.seed}\NormalTok{(}\DecValTok{1}\NormalTok{)}
\end{Highlighting}
\end{Shaded}

\begin{Shaded}
\begin{Highlighting}[]
\CommentTok{#No. of values (n) = 40, lambda = 0.2, No.of iterations, at least 1000, numsim=2000,Theoretical mean =1 / lambda or 1 / 0.2 }
\NormalTok{lambda <-}\StringTok{ }\FloatTok{0.2} 
\NormalTok{nosim <-}\StringTok{ }\DecValTok{1}\OperatorTok{:}\DecValTok{1000} \CommentTok{# Number of Simulations/rows}
\NormalTok{n <-}\StringTok{ }\DecValTok{40} 
\end{Highlighting}
\end{Shaded}

\subsubsection{Generating data using
rexp}\label{generating-data-using-rexp}

Use the rexp function to develop a dataset with the mean and lambda
specified above.

\begin{Shaded}
\begin{Highlighting}[]
\CommentTok{#sd(apply(matrix(rnorm(nosim*n), nosim), 1, mean))}

\CommentTok{#Create a matrix of simulated values:}
\NormalTok{e_matrix <-}\StringTok{ }\KeywordTok{data.frame}\NormalTok{(}\DataTypeTok{x =} \KeywordTok{sapply}\NormalTok{(nosim, }\ControlFlowTok{function}\NormalTok{(x) \{}\KeywordTok{mean}\NormalTok{(}\KeywordTok{rexp}\NormalTok{(n, lambda))\}))}
\end{Highlighting}
\end{Shaded}

\textbf{1.} Show where the distribution is centered at and compare it to
the theoretical center of the distribution.

\begin{Shaded}
\begin{Highlighting}[]
\NormalTok{sim_mean <-}\StringTok{ }\KeywordTok{apply}\NormalTok{(e_matrix, }\DecValTok{2}\NormalTok{, mean)}
\NormalTok{sim_mean}
\end{Highlighting}
\end{Shaded}

\begin{verbatim}
##        x 
## 4.990025
\end{verbatim}

Which is very close to the expected theoretical center of the
distribution:

\begin{Shaded}
\begin{Highlighting}[]
\NormalTok{th_mean <-}\StringTok{ }\DecValTok{1}\OperatorTok{/}\NormalTok{lambda}
\NormalTok{th_mean}
\end{Highlighting}
\end{Shaded}

\begin{verbatim}
## [1] 5
\end{verbatim}

\textbf{2.}Show how variable it is and compare it to the theoretical
variance of the distribution. .

\begin{Shaded}
\begin{Highlighting}[]
\NormalTok{sim_SD <-}\StringTok{ }\KeywordTok{sd}\NormalTok{((e_matrix}\OperatorTok{$}\NormalTok{x)) }
\NormalTok{sim_SD}
\end{Highlighting}
\end{Shaded}

\begin{verbatim}
## [1] 0.7817394
\end{verbatim}

\begin{Shaded}
\begin{Highlighting}[]
\NormalTok{sim_Var <-}\StringTok{ }\KeywordTok{var}\NormalTok{(e_matrix}\OperatorTok{$}\NormalTok{x)}
\NormalTok{sim_Var}
\end{Highlighting}
\end{Shaded}

\begin{verbatim}
## [1] 0.6111165
\end{verbatim}

Let's compare, the expected theretical SD and Variance are:

\begin{Shaded}
\begin{Highlighting}[]
\NormalTok{th_SD <-}\StringTok{ }\NormalTok{(}\DecValTok{1}\OperatorTok{/}\NormalTok{lambda)}\OperatorTok{/}\KeywordTok{sqrt}\NormalTok{(n)}
\NormalTok{th_SD}
\end{Highlighting}
\end{Shaded}

\begin{verbatim}
## [1] 0.7905694
\end{verbatim}

\begin{Shaded}
\begin{Highlighting}[]
\NormalTok{th_Var <-}\StringTok{ }\NormalTok{th_SD}\OperatorTok{^}\DecValTok{2}
\NormalTok{th_Var}
\end{Highlighting}
\end{Shaded}

\begin{verbatim}
## [1] 0.625
\end{verbatim}

Comparing Theoretical and actual Values of mean,Standard deviation and
variance Table

\begin{longtable}[]{@{}lll@{}}
\toprule
Variable & Theoretical val & Actual Val\tabularnewline
\midrule
\endhead
Mean & 5 & 5.048\tabularnewline
SD & 0.791 & 0.796\tabularnewline
Var & 0.625 & 0.634\tabularnewline
\bottomrule
\end{longtable}

We can verify that the differences are minimal, as expected.

\textbf{3.} Show that the distribution is approximately normal.

\begin{Shaded}
\begin{Highlighting}[]
\NormalTok{plot <-}\StringTok{ }\KeywordTok{ggplot}\NormalTok{(}\DataTypeTok{data =}\NormalTok{ e_matrix, }\KeywordTok{aes}\NormalTok{(}\DataTypeTok{x =}\NormalTok{ x)) }\OperatorTok{+}\StringTok{ }
\StringTok{    }\KeywordTok{geom_histogram}\NormalTok{(}\KeywordTok{aes}\NormalTok{(}\DataTypeTok{y=}\NormalTok{..density..), }\DataTypeTok{binwidth =} \FloatTok{0.20}\NormalTok{, }\DataTypeTok{fill=}\StringTok{"slategray3"}\NormalTok{, }\DataTypeTok{col=}\StringTok{"black"}\NormalTok{)}
\NormalTok{plot <-}\StringTok{ }\NormalTok{plot }\OperatorTok{+}\StringTok{ }\KeywordTok{labs}\NormalTok{(}\DataTypeTok{title=}\StringTok{"Density of 40 Numbers from Exponential Distribution"}\NormalTok{, }\DataTypeTok{x=}\StringTok{"Mean of 40 Selections"}\NormalTok{, }\DataTypeTok{y=}\StringTok{"Density"}\NormalTok{)}
\NormalTok{plot <-}\StringTok{ }\NormalTok{plot }\OperatorTok{+}\StringTok{ }\KeywordTok{geom_vline}\NormalTok{(}\DataTypeTok{xintercept=}\NormalTok{sim_mean,}\DataTypeTok{size=}\FloatTok{1.0}\NormalTok{, }\DataTypeTok{color=}\StringTok{"black"}\NormalTok{)}
\NormalTok{plot <-}\StringTok{ }\NormalTok{plot }\OperatorTok{+}\StringTok{ }\KeywordTok{stat_function}\NormalTok{(}\DataTypeTok{fun=}\NormalTok{dnorm,}\DataTypeTok{args=}\KeywordTok{list}\NormalTok{(}\DataTypeTok{mean=}\NormalTok{sim_mean, }\DataTypeTok{sd=}\NormalTok{sim_SD),}\DataTypeTok{color =} \StringTok{"dodgerblue4"}\NormalTok{, }\DataTypeTok{size =} \FloatTok{1.0}\NormalTok{)}
\NormalTok{plot <-}\StringTok{ }\NormalTok{plot}\OperatorTok{+}\StringTok{ }\KeywordTok{geom_vline}\NormalTok{(}\DataTypeTok{xintercept=}\NormalTok{th_mean,}\DataTypeTok{size=}\FloatTok{1.0}\NormalTok{,}\DataTypeTok{color=}\StringTok{"indianred4"}\NormalTok{,}\DataTypeTok{linetype =} \StringTok{"longdash"}\NormalTok{)}
\NormalTok{plot <-}\StringTok{ }\NormalTok{plot }\OperatorTok{+}\StringTok{ }\KeywordTok{stat_function}\NormalTok{(}\DataTypeTok{fun=}\NormalTok{dnorm,}\DataTypeTok{args=}\KeywordTok{list}\NormalTok{(}\DataTypeTok{mean=}\NormalTok{th_mean, }\DataTypeTok{sd=}\NormalTok{th_SD),}\DataTypeTok{color =} \StringTok{"darkmagenta"}\NormalTok{, }\DataTypeTok{size =} \FloatTok{1.0}\NormalTok{)}
\NormalTok{plot}
\end{Highlighting}
\end{Shaded}

\includegraphics{Final-project_files/figure-latex/unnamed-chunk-12-1.pdf}

we conclude that the function appears to aproximate to nearly Normal.

\end{document}
